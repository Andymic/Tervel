\documentclass{article}

\begin{document}

\title{An Efficient Lock-free Logarithmic Search Data Structure Based on Multi-dimensional List}

\maketitle

A \emph{dictionary} abstract data type stores a set of key-value pairs where the keys are drawn from a totally ordered universe. 
It is defined by its semantics that specify three canonical operations: \textsc{Insert} which adds a key-value pair, \textsc{Delete} which removes the value associated with a specific key, and \textsc{Find} which returns the associated value if the specific key is in the set.
Binary search trees are among the most ubiquitous sequential data structures for implementing dictionaries. 
However, due to complicated tree rotation operations, designing an efficient self-balancing non-blocking BST is very challenging and remains an active research topic~\cite{brown2014general,natarajan2014fast,ellen2014amortized}.
In recent research studies~\cite{linden2013skiplist,sundell2004scalable,fraser2004practical}, non-blocking dictionaries based on skiplists are gaining momentum.
A skiplist~\cite{pugh1990skip} eliminates the need of rebalancing by using several linked lists, organized into multiple levels, where each list skips a few elements. 
However, write operations involve updating shortcut links in distant nodes, which incurs unnecessary data dependencies among concurrent operations and limits the overall throughput.

In this work, the PI presents a linearizable lock-free dictionary implementation based on a multi-dimensional list (MDList) using only single-word \textsc{CompareAndSwap} (CAS) primitive.
The core idea is to partition a linked list into shorter lists and rearrange them in a multi-dimensional space to facilitate search.
Nodes in the proposed dictionary are ordered by coordinate prefix, which eliminates the need of rebalancing and randomization.
Its physical layout is deterministic and independent of execution histories, which provides a unique location for each key, and simplifies the \textsc{Find} algorithm.
Each insertion or deletion modifies at most two consecutive nodes, which allows concurrent updates to be executed with minimal interference.

The search operation examines one coordinate at a time and locates correspondent partitions by traversing nodes that belong to each dimension. 
The \textsc{Insert} operation involves two steps: node splicing and child adoption.
In the first step, the appropriate insert position of the new node is determined by predecessor query.
Splicing involves pointing to the old child of the predecessor from the new node and updating the child pointer of the predecessor atomically using CAS.
The second step is needed when the dimension of the old child has been changed due to the insertion of a new node.
This extra child adoption process ensures that the nodes which are no longer accessible through the old child can be reached through the new node.
This step is atomically announced in a adoption descriptor so that any thread that attempt to read an inaccessible node pointer must help the newly inserted node to finish its child adoption.

Normally, deletion and insertion operation works reciprocally: the former may promote a node to a lower dimension while the latter may demote a node to a higher dimension.
This works fine for sequential algorithms, but in concurrent execution where threads help each other, bidirectional change of nodes' dimension incurs contention and synchronization issues.
Consider a deletion operation that tries to promote a node's dimensionality, it may encounter some unfinished helping threads that are simultaneously demoting the node.
This causes access conflict on the shared data. 
Additional synchronization is required to prevent threads from interfering with each other when they perform helping task on the same node.
The lock-free \textsc{Delete} operation is designed to be asymmetrical in the sense that it dose not remove any node from the data structure.
It only marks a node for logical deletion~\cite{harris2001pragmatic}, and leaves the execution of physical deletion to subsequent \textsc{Insert} operations.
When a new node ($n_n$) is inserted immediately before a logically deleted node ($n_d$), $n_n$ expunge $n_d$ from the data structure by skipping $n_n$ and linking directly into the child nodes of $n_d$.
Since the physical deletion is embedded in the insertion operation, we reduce the interaction between insertion and deletion operations to a minimal and achieved an overall simple design of lock-free dictionary.
Since a logically deleted node only gets purged when an insertion take place immediately in front of it, there will be a number of zombie nodes.
On the other hand, there is less chance that physical deletions can interfere with concurrent insertion.
We thus trade memory footprint for scalability.

In our experimental evaluation, we compare our algorithm with the best available skiplist-based~\cite{dick2014logarithmic,fraser2004practical} and BST-based dictionaries~\cite{ellen2014amortized,arbel2014concurrent,bronson2010practical} using a micro-benchmark on a 64-core NUMA system and a 12-core SMP system.
The result shows that on average our algorithm outperforms the alternative approaches by $30\%$.
It exhibits excellent scalability by obtaining as much as $100\%$ speedup over the best alternatives under high levels of concurrency with large key spaces.
We also show that the dimensionality of an MDList-based dictionary can be tuned to fit different application scenarios: a high-dimensional dictionary behaves more like a tree and speeds up insertions; a low-dimensional priority queue behaves more like an ordered linked list and speeds up deletions.

\bibliographystyle{abbrv}
\bibliography{citation}

\end{document}
